%-----------------------------------------------------------------------------
%
%               Template for sigplanconf LaTeX Class
%
% Name:         sigplanconf-template.tex
%
% Purpose:      A template for sigplanconf.cls, which is a LaTeX 2e class
%               file for SIGPLAN conference proceedings.
%
% Guide:        Refer to "Author's Guide to the ACM SIGPLAN Class,"
%               sigplanconf-guide.pdf
%
% Author:       Paul C. Anagnostopoulos
%               Windfall Software
%               978 371-2316
%               paul@windfall.com
%
% Created:      15 February 2005
%
%-----------------------------------------------------------------------------


\documentclass{sigplanconf}

% The following \documentclass options may be useful:

% preprint      Remove this option only once the paper is in final form.
% 10pt          To set in 10-point type instead of 9-point.
% 11pt          To set in 11-point type instead of 9-point.
% authoryear    To obtain author/year citation style instead of numeric.

\usepackage{amsmath}
\usepackage{todonotes}
\newcommand{\mytodo}[2][]{\todo[color=gray!20,size=\scriptsize,fancyline,#1]{#2}}
\newcommand{\redtodo}[2][]{\todo[color=red!20,size=\scriptsize,fancyline,#1]{#2}}

\begin{document}

\special{papersize=8.5in,11in}
\setlength{\pdfpageheight}{\paperheight}
\setlength{\pdfpagewidth}{\paperwidth}

\conferenceinfo{ICFP '15}{September 1--3, 2015, Vancouver, British Columbia, Canada}
\copyrightyear{2015}
%% camera-ready data
\copyrightdata{978-1-nnnn-nnnn-n/yy/mm}
\doi{nnnnnnn.nnnnnnn}

% Uncomment one of the following two, if you are not going for the
% traditional copyright transfer agreement.

%\exclusivelicense                % ACM gets exclusive license to publish,
                                  % you retain copyright

%\permissiontopublish             % ACM gets nonexclusive license to publish
                                  % (paid open-access papers,
                                  % short abstracts)

\titlebanner{banner above paper title}        % These are ignored unless
\preprintfooter{short description of paper}   % 'preprint' option specified.

\title{Total (Co)Programming with Guarded Recursion}
%%\subtitle{Subtitle Text, if any}

\authorinfo{Andrea Vezzosi}
           {Chalmers Univerisity of Technology}
           {vezzosi@chalmers.se}
%% \authorinfo{Name2\and Name3}
%%            {Affiliation2/3}
%%            {Email2/3}

\maketitle

\begin{abstract}
In total functional (co)programming valid programs are guaranteed to
always produce (part of) their output in a finite number of steps.

Enforcing this property while not sacrificing expressivity has been
challenging, traditionally systems like Agda and Coq have relied on a
syntactic restriction on (co)recursive calls, which is inherently
anti-modular and sometimes leads to unexpected interactions with new
features.

Guarded recursion, first introduced by Nakano, has been recently
applied in the coprogramming case to ensure totality through typing
instead, by capturing through a new modality the relationship between
the consumption and the production of data.

In this paper we show how that approach can be extended to
additionally ensure termination of recursive programs, through the
introduction of a dual modality and the interaction between the
two. As in the coprogramming case, however, we need clock variables to
separate different flows of data. Existential quantification over
clocks then recovers inductive types from guarded ones, dually to how
universal quantification recovers coinductive types.

%% - guarded recursion
%% - new modality
%% - existential

\end{abstract}

\category{CR-number}{subcategory}{third-level}

% general terms are not compulsory anymore,
% you may leave them out
\terms
term1, term2

\keywords
keyword1, keyword2

\section{Introduction}
To get a recursive program to pass the standard termination checkers
of systems like Coq or Agda, the recursive calls must have
structurally smaller arguments. This is the case for the $map$ function on lists.
\begin{verbatim}
map : ∀ {A B : Set} → (A → B) → List A → List B
map f []       = []
map f (x ∷ xs) = f x ∷ map f xs
\end{verbatim}
Here the argument $xs$ is manifestly smaller than $x ∷ xs$ because
it's a direct ``subtree''.

However such a syntactic check gets in the way of code reuse and
compositionality, especially when trying to use higher-order
combinators.

\begin{verbatim}
data RoseTree a = Node (List (RoseTree a))

mapRT : ∀ {A B : Set} → (A → B) → RoseTree A → RoseTree B
mapRT f (Node ts) = map (\ t -> mapRT f t) ts
\end{verbatim}
The definition of mapRT is not accepted because syntactically $t$ has
no relation to $Node ts$, we need to pay attention to the semantics of
$map$ to accept this definition as terminating, in fact a common
workaround is to essentially inline the code for $map$.
\begin{verbatim}
mapRT : ∀ {A B : Set} → (A → B) → RoseTree A → RoseTree B
mapRT f (Node ts) = mapmapRT f ts
  where
    mapmapRT f []       = []
    mapmapRT f (t ∷ ts) = mapRT f t ∷ mapmapRT f ts    
\end{verbatim}

Here we have spelled out the nested recursion as a function
$mapmapRT$, and the checker can figure out that $mapRT$ is only going
from the call $mapRT f (Node (t ∷ ts))$ to $mapRT f t$, so it's accepted.

Such a system then actively fights abstraction, and offers little
appeals other than sticking to some specific ``design patterns'' for
(co)recursive definitions. From that comes the temptation to put more
sophisticated heuristics into the checker, which from time to time
need to be retracted.\mytodo{expand on prolem with univalence? at least add citations}

Recently there's been a fair amount of research into moving to the
type level the information about how functions consume and produce
data, so that totality checking is reduced to type-checking
\cite{BobConor} \cite{Rasmus} or close to it \cite{Andreas}, \cite{Foundational Coinduction}.

This paper extends to the case of recursion
the previous work on Guarded Recursion \cite{BobConor} \cite{Rasmus},
which was focused on corecursion and where inductive types were only
given their primitive recursor, by presenting a more relaxed model where
the connectives for guarded recursion can convive with their duals.

We make the following contributions:
\begin{itemize}
\item We define a core type system which supports total programming by
  combining guarded recursion, both existential and universal clock
  quantification and a new delay modality. From these primitives we
  can derive both (finitely branching) initial algebras and final
  coalgebras, i.e. inductive and coinductive datatypes. The type
  system is an extension of Martin Loef Type Theory.
\item We give an interpretation of the system into a relationally
  parametric model of dependent type theory \cite{atkey:param-model},
  drawing a parallel between clock quantification and parametric
  polymorphism.
\item In particular we show how representationally independent
  existential types can be defined in a predicative way by truncating
  sigma types to the universe of discrete reflexive graphs.
\end{itemize}


%% - Describe Problem
%%    - problem 1: replace syntactic checks
%%            - examples of anti-modularity? 
%%               - rose tree
%%               - div for later
%%            - inconsistencies? with univalence i mean 
%%            - too much stuff? Write it, can be moved.
%%    - problem 2: guarded recursion applied only to corecursion
%%       - coinductive types defined with guards and fixpoints but inductive types as primitive
      
%% - List the contributions
%%   - dualize! (find better name than codelay)
%%   - allows codelay and existential -- why no codelay and pure for delay?
%%   - extension on MLTT with delay and codelay mod, fixpoints on universes, quantifiers over clock variables.
%%   - denotational model based on logical relations/parametricity
%%   - existential quant. as representationally independent one via relational truncation
%%  \ref{termination-univ-Agda} \ref{termination-univ-Coq}

%% \Section{2}
%%   - guarded rose-tree example, don't show fix
%%   - how do you really write introductions for e.g. nil, or zero,  oh actually (A ~ ∃ k, A) when k free in A
%%   - dualize! but 
%%   - why different semantics, ``i don't like non-termination''
%%   - comparing
%%      - model makes us give up general pure : a -> delay a
%%      - example hard to do with sized types? mostly reasoning, or for coinduction something from the HOL paper
%%                                          also hard to ``forget'' sizes (find example?)
%%                                          - recognizers! Recognizers.agda
%%   - try to define a universe ala induction-recursion. ala guardedness-preserving, see release notes. 

The text of the paper begins here.

Lots of text.

More text.

Lots of text.

More text.


Lots of text.

More text.

Lots of text.

More text.


Lots of text.

More text.

Lots of text.

More text.

Lots of text.

More text.

Lots of text.

More text.

Lots of text.

More text.

Lots of text.

More text.

Lots of text.

More text.

Lots of text.

More text.

Lots of text.

More text.

Lots of text.

More text.


Lots of text.

More text.

Lots of text.

More text.


Lots of text.

More text.

Lots of text.

More text.

Lots of text.

More text.

Lots of text.

More text.

Lots of text.

More text.

Lots of text.

More text.

Lots of text.

More text.

Lots of text.

More text.


Lots of text.

More text.

Lots of text.

More text.




Lots of text.

More text.

Lots of text.

More text.

Lots of text.


Lots of text.

More text.

Lots of text.

More text.

Lots of text.

More text.

Lots of text.

More text.

Lots of text.

More text.

Lots of text.

More text.

Lots of text.

More text.

Lots of text.

More text.

Lots of text.

More text.

Lots of text.

More text.

Lots of text.

More text.

Lots of text.

More text.

Lots of text.

\appendix
\section{Appendix Title}

This is the text of the appendix, if you need one.

\acks

Acknowledgments, if needed.

% We recommend abbrvnat bibliography style.

\bibliographystyle{abbrvnat}

% The bibliography should be embedded for final submission.

\begin{thebibliography}{}
\softraggedright

\bibitem[Smith et~al.(2009)Smith, Jones]{smith02}
P. Q. Smith, and X. Y. Jones. ...reference text...

\end{thebibliography}


\end{document}

%                       Revision History
%                       -------- -------
%  Date         Person  Ver.    Change
%  ----         ------  ----    ------

%  2013.06.29   TU      0.1--4  comments on permission/copyright notices
